\documentclass[11pt,a4paper]{article}
\usepackage{amsthm}
\usepackage{bbm}
\usepackage{mathrsfs}
\usepackage{amsfonts} %%% i.e. use 12pt type
\usepackage[dvipsnames,usenames]{color}
\usepackage{graphicx,latexsym,bm,amsmath,amssymb,verbatim,multicol,lscape}
\usepackage{enumerate}
\usepackage {cancel}
\usepackage[utf8]{inputenc}
\usepackage{amsmath}
\usepackage{amsfonts}
\usepackage{amssymb}
\usepackage{graphicx}
\usepackage{cite}
\usepackage{hyperref}



\usepackage{graphicx,latexsym,bm,amsmath,amssymb,verbatim,multicol,lscape}
\usepackage{enumerate}
\usepackage {cancel}
% ----------------------------------------------------------------
\vfuzz2pt % Don't report over-full v-boxes if over-edge is small
\hfuzz2pt % Don't report over-full h-boxes if over-edge is small
% THEOREMS -------------------------------------------------------
\newtheorem{thm}{Theorem} [section]
\newtheorem{lem}{Lemma}[section]
\newtheorem{pro}{Proposition}[section]
\newtheorem{cor}{Corollary}[section]
\newtheorem{ass}{Assumption}[section]
\newtheorem{exm}{Example}[section]
\theoremstyle{definition}
\newtheorem{defn}{Definition}[section]
\theoremstyle{remark}
\newtheorem{rem}{Remark}[section]
\newtheorem{con}[thm]{Conjecture}
\numberwithin{equation}{section}
\DeclareMathOperator{\Tr}{Tr}
\DeclareMathOperator{\ind}{ind}
\DeclareMathOperator{\rank}{rank}
\allowdisplaybreaks
% MATH -----------------------------------------------------------
\newcommand{\norm}[1]{\left\Vert#1\right\Vert}
\newcommand{\abs}[1]{\left\vert#1\right\vert}
\newcommand{\set}[1]{\left\lbrace #1 \right\rbrace}
\newcommand{\p}[1]{\left( #1 \right)}
\newcommand{\Real}{\mathbb R}
\newcommand{\Natural}{\mathbb N}
\newcommand{\eps}{\varepsilon}
\newcommand{\To}{\longrightarrow}
\newcommand{\BX}{\mathbf{B}(X)}
\newcommand{\A}{\mathcal{A}}
\newcommand{\SSS}{\stackrel}



\title{Signal-to-Computation Ratio: A Measure of System Complexity and Information Extraction}
\author{J.L}
\date{\today}



%%%%%%%%%%%%%%%%%%%%%%%%%%%%%%%%%%%%%%%%%%%%%%%%%%%%%%%%%%%%%%%%%%%%%%%%%%%%%%%%%%%%%%%%%%%%%%%%%%%



\begin{document}

\maketitle

\begin{abstract}
In this paper, we introduce the concept of the Signal-to-Computation Ratio (SCR), a new metric for evaluating the computational efficiency required to extract information from a system. SCR is an intrinsic property of the system and is independent of the algorithm used. We define SCR mathematically and extend the concept to time-evolving dynamical systems.
\end{abstract}

\section{Introduction}
In many complex systems, the efficiency of extracting information from a system is of paramount importance. 
While the signal-to-noise ratio (SNR) is a well-known metric in communication theory, we propose an analogous concept, the Signal-to-Computation Ratio (SCR), to measure the computational cost relative to the amount of information extracted from a system, if at-all possible. 
The SCR is a system-specific property, reflecting how "difficult" it is to extract meaningful information from that system, irrespective of the specific algorithms used.

The goal of this paper is to formalize SCR, explore its properties, and examine its applications.
Furthermore, we investigate it  in the context of time-evolving dynamical systems. 

\section{Definitions}

\subsection{System}

The system is defined as follows:

\[
X := \{ x_t \}_{t \in T}
\]

where $T$ is the index set.
This family represents the various system states, indexed by $t\in T$, 
$t$ might represent time or any other indexing parameter.

\subsection{Binary Query}

We define the binary set $\mathbb{B}$ and the set of binary queries $Q$ as follows:

\[
\mathbb{B} := \{ 0, 1 \}
\]
\[
Q := \{ q : X \rightarrow \mathbb{B} \}
\]

Two binary queries $q$ and $q'$ are considered identical if they return the same output for all elements of $X$:

\[
\forall q, q' \in Q, \quad q = q' \iff \forall x_t \in X, q(x) = q'(x)
\]

Furthermore, two queries $q$ and $q'$ are considered equivalent under a transformation $f$ from the set $\amalg$ of invertible functions:

\[
\forall q, q' \in Q, \quad q \sim q' \iff \exists f \in \amalg, \forall x_t \in X, f \circ q(x_t) = q'(x_t)
\]

\subsection{Horizon}

The horizon $H$ is defined as the set of functions that map elements of $X$ to data:

\[
H = \{ h : X \rightarrow D \}
\]

Each $h \in H$ reflects the information available about $x \in X$.

\subsection{Sandbox}

The sandbox $S$ is defined as the Cartesian product of $X$ and $H$:

\[
S := X \times H
\]

In this framework, instead of writing $(x_t, h) \in S$, we use the notation $x_t | h \in S$.

\subsection{Binary Algorithm}

A binary algorithm processes some data $d \in D$ and outputs a single bit:

\[
A := \{ a : D \rightarrow \mathbb{B} \}
\]

For $a \in A$, $q \in Q$, and $h \in H$, the relationship between an algorithm and a query is defined as:

\[
a \underset{h}{\rightsquigarrow} q \iff \forall x_t \in X, q(x_t) = a(h(x_t))
\]

Two binary algorithms $a$ and $a'$ are considered equivalent under $h$ if they produce equivalent queries:

\[
a \underset{h}{\sim} a' \iff \exists q \in Q , a \underset{h}{\rightsquigarrow}q \wedge a' \underset{h}{\rightsquigarrow}q
\]

The set of algorithms corresponding to a query $q$ under horizon $h$ is denoted as:

\[
A_{q|h} := \{ a \in A : a \underset{h}{\rightsquigarrow} q \}
\]

\subsection{Meaningful Questions Set}

The set of meaningful questions under horizon $h$ is defined as:

\[
Q^*_h := \left\{ q \in \frac{Q}{\sim} : \exists a \in A , a \underset{h}{\rightsquigarrow} q \right\}
\]

\subsection{Computation Resource}

The computation resource cost $\mathfrak{c}(a,d)$ is the cost of computing $a(d)$:

\[
\mathfrak{c} : A \times D \rightarrow \mathbb{R}^+
\]

The computation resource cost associated with a question $q$, given sandbox $x_t|h$, is:

\[
c : S \rightarrow (Q \rightarrow \mathbb{R}^+)
\]
\[
c_{x_t|h}(q) := \min_{a \in A_{q|h}} \mathfrak{c}(a, h(x_t))
\]

\subsection{Extraction Cost}

The extraction cost $e_{x_t|h}(n)$ is the total computation resource cost to extract $n$ bits of information from the sandbox $x_t|h$:

\[
e : S \rightarrow (\mathbb{N} \rightarrow \mathbb{R}^+)
\]
\[
e_{x_t|h}(n) := \min_{\left\{ \Omega \subseteq Q^*_h : |\Omega| = n \right\}} \sum_{q \in \Omega} c_{x_t|h}(q)
\]

\subsection{System Computational Resource (SCR)}

The System Computational Resource (SCR) is defined as:

\[
SCR : S \rightarrow (\mathbb{N} \rightarrow \mathbb{R}^+)
\]
\[
SCR_{x_t|h}(n) := \frac{n}{e_{x_t|h}(n)}
\]

\section{Application to Dynamical Systems}

Let $T = \mathbb{R}$ represent time. The horizon over a time interval $[t_0, t_1]$ is:

\[
h_{[t_0, t_1]} := \{ x_t : t \in [t_0, t_1] \}
\]

The System Computational Resource in a dynamic system is then:

\[
SCR_{x_{t+\Delta t} | h_{[0,t]}}(n)
\]

\section{Conclusion}

This framework provides a formalized approach to understanding binary queries, computational resource costs, and system efficiency. The System Computational Resource (SCR) metric offers a means to evaluate the efficiency of computational processes, particularly in dynamic systems. Future work could explore specific applications of this framework in areas such as optimization, machine learning, and automated decision-making.

\bibliographystyle{plain}
\bibliography{references}

\end{document}
