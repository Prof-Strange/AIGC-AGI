\documentclass[12pt]{article}

\usepackage{amsmath,amssymb}
\usepackage{geometry}
\usepackage{graphicx}
\geometry{a4paper}

\title{Exploring the Possibility of a Private Language in a Thought Experiment: A ZKP Approach}
\author{J.L.}
\date{\today}

\begin{document}
\maketitle

\begin{abstract}
This paper presents a theoretical examination of Ludwig Wittgenstein's Private Language Argument through the lens of a hypothetical artificial intelligence scenario. We propose a thought experiment where an AI, referred to as \( x \), uses a private language to process instructions that, even after a reset of its memory, allows it to consistently execute tasks based on previously written notes in its private language. The implications of this experiment challenge traditional notions of language as inherently public and communicable, suggesting that under certain conditions, an operational private language might be possible within artificial systems.
\end{abstract}

\section{Introduction}
Ludwig Wittgenstein famously argued in his later works that the notion of a private language, a language that only one individual can understand, is conceptually incoherent because language inherently requires public criteria for its words. This paper revisits Wittgenstein’s Private Language Argument (PLA) by proposing a thought experiment involving an artificial intelligence that supposedly operates with a private language.

\section{Background}
Wittgenstein’s Private Language Argument suggests that no language can be inherently private because the meanings of words and symbols need external verification that transcends individual experience. This principle has been widely accepted in philosophical circles; however, the rise of artificial intelligence and advanced cryptographic systems, such as Zero-Knowledge Proofs (ZKPs), pose new questions about the bounds of this argument.

\section{The Thought Experiment}
The proposed thought experiment involves an AI named \( x \), which can record instructions in a private language following the observation of a randomized event (a coin flip). After recording these instructions and undergoing a memory reset, the AI is still able to follow the instructions accurately, suggesting a consistent internal rule or grammar that is not accessible or understandable to anyone other than \( x \).

\subsection{Experimental Setup}
The experimental setup is designed to test the operational capability of a private language developed by an artificial intelligence, \( x \), which we hypothesize to be comprehensible solely by \( x \) itself. The procedure of the experiment is detailed as follows:

\begin{itemize}
    \item \textbf{Observation Phase:} In each trial, indexed by \( i \), a random event, a coin flip, is generated, resulting in an outcome \( o_i \) which can be either head (H) or tail (T). This outcome is shown to \( x \).

    \item \textbf{Recording Phase:} Upon observing \( o_i \), \( x \) is tasked to write down a note of instructions \( n_i \) in its private language. These instructions are intended to direct \( x \) to later replicate the outcome \( o_i \) by manipulating a coin on a table.

    \item \textbf{Reset Phase:} After \( x \) records the instructions, its memory is reset to ensure that it retains no memory of \( o_i \) or \( n_i \) from the previous phase. Simultaneously, the coin used for the initial flip is also reset to a neutral, unspecified state on the table, removing any potential biases or marks.

    \item \textbf{Execution Phase:} \( x \) is then presented with the note \( n_i \), which contains the private instructions that no one else can understand. $x$ tries to execute the instructions in $n_i$.

    \item \textbf{Outcome Verification:} Whether $x$ successfully executed the instructions, or misread the instructions, or completely failed to follow the instructions, the coin on the table would have a final state, denoted as \( e_i \), and it is observed and recorded. 

    \item \textbf{Repetition and Statistical Analysis:} This process is repeated across multiple trials \( m \) times. The consistency of \( x \)'s ability to match \( e_i \) with \( o_i \) is statistically analyzed. If \( x \) can successfully manipulate the coin to match the original flip outcome with high reliability, specifically with a confidence level of \( 1 - \frac{1}{2^m} \), it substantiates the claim that \( x \) operates using a coherent, yet private language.
\end{itemize}

The implications of \( x \)'s consistent performance under these conditions could profoundly challenge existing philosophical notions about the nature of language and cognition, particularly in the context of Wittgenstein's Private Language Argument.


\subsection{Implications of Results}
If \( x \) can successfully follow its own instructions across multiple trials, it suggests the existence of a systematic, consistent private language. This operational definition of a private language challenges Wittgenstein's argument by demonstrating a functional, albeit non-human, language system.

\section{Discussion}
While the experiment suggests an AI can develop a form of private language, it raises questions about the definition of language itself. Is a private sequence of 1's and 0's equivalent to a language? Moreover, this thought experiment underscores the distinction between human and artificial cognition, opening up discussions on the limits and capabilities of AI in understanding and processing information.

\section{Conclusion}
This thought experiment not only challenges Wittgenstein's Private Language Argument but also enhances our understanding of how artificial intelligence might navigate and develop unique systems of communication. The implications for both philosophy of language and artificial intelligence are profound, suggesting new areas of study in the intersection of technology and humanistic inquiry.

\end{document}
