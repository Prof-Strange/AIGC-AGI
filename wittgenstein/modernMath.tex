\documentclass{article}
\usepackage[margin=1in]{geometry}
\usepackage{amsmath}
\usepackage{amsfonts}
\usepackage{amssymb}
\usepackage{hyperref}

\title{Without Machine-Aided Proofs, Modern Pure (Noninterdisciplinary) Mathematics is a Private Language of a Dissociative Meta-Subject: A Close Corollary of Wittgensteinian Private Language Argument}
\author{J.L.}
\date{November 19, 2025}

\begin{document}

\maketitle

\begin{abstract}
This paper explores a radical extension of Ludwig Wittgenstein's private language argument (PLA) to the strictly noninterdisciplinary subset of modern pure mathematics---that is, mathematics deliberately decoupled from empirical science. We argue that, in the absence of machine-aided proof verification (computational grounding), this domain devolves into an ungrounded private language akin to the internal discourse of a single subject afflicted with Dissociative Identity Disorder (DID). By conceptualizing the global community of pure mathematicians as a single ``Epistemic Meta-Subject,'' we demonstrate that inter-mathematician communication (peer review) functions merely as intra-subjective dialogue between distinct alters. Without the external, rigid criteria provided by physical implementation or computational verification, the complexity of modern proofs renders them susceptible to the core critique of the PLA: the collapse of the distinction between following a rule and merely thinking one is following a rule. Consequently, without computational aids, noninterdisciplinary pure mathematics risks becoming a semantic closed loop, devoid of external truth conditions.
\end{abstract}

\section{Introduction}
Ludwig Wittgenstein's private language argument, centrally located in his \emph{Philosophical Investigations} \cite{wittgenstein1953}, posits the impossibility of a language which is in principle intelligible only to a single user. The crux of the argument lies in the necessity of a distinct criterion for correctness: for a sign to have meaning, there must be a distinction between correct usage and the mere impression of correct usage.

In this paper, we extend the PLA to a specific, high-altitude stratum of modern intellectual inquiry: pure mathematics that is explicitly noninterdisciplinary. This refers to branches concerned with abstract structures developed without reference to empirical sciences or external applications (e.g., esoteric regions of category theory, topology, or abstract algebra pursued solely for intrinsic interest). We contend that, absent machine-aided proofs---hypothetically assuming a world without computational verification---this noninterdisciplinary mathematics functionally resembles a private language.

We introduce the concept of the ``Epistemic Meta-Subject'': the collective set of all pure mathematicians working in this decoupled mode, viewed not as a community of distinct interlocutors, but as a unified entity with multiple ``personalities'' (individual mathematicians). Using the structural analogy of Dissociative Identity Disorder (DID), we argue that peer review in this closed system lacks the status of public verification. Instead, it resembles the internal negotiation between alters in a fragmented mind. Historically, pre-Newtonian mathematics was anchored in the natural world. The post-Hilbertian decoupling of mathematics from nature, if combined with a lack of computational verification, removes the final external check, rendering the language private and, in Wittgensteinian terms, strictly meaningless.

\section{The Private Language Argument and the Criterion of Correctness}
The standard interpretation of Wittgenstein challenges the Cartesian notion that mental states (and the terms referring to them) are self-validating. The essential requirement for language is normativity: rules must exist, and rules require a standard of correctness independent of the rule-follower.

\begin{itemize}
    \item \textbf{The Memory Problem:} If an individual attempts to define a sign ``S'' for a private sensation, they rely solely on memory to identify ``S'' in the future. Wittgenstein argues this is akin to a man buying several copies of the morning newspaper to assure himself that what it said was true.
    \item \textbf{The Public Solution:} Typically, language is rescued from privacy by the community. The intersubjective consensus provides the external check.
\end{itemize}

However, this paper interrogates the ``public'' nature of the mathematical community itself. In elementary mathematics, the physical world (counting apples) provides an external check. In advanced, noninterdisciplinary pure mathematics, the theorems---such as the Classification of Finite Simple Groups (the Enormous Theorem)---involve proofs of such profound complexity that they defy unaided individual verification \cite{gorenstein1994}. If the ``community'' is hermetically sealed from the physical world, does it constitute a valid ``public,'' or merely a complex, collective solipsism?

\section{The Decoupling from Empirical Reality}
Historically, mathematics was the abstraction of physical observation. The geometry of Euclid and the calculus of Newton were constrained by the need to describe physical space and motion. This provided an ontological anchor: if the math was wrong, the bridge fell down or the planet was not where predicted.

Modernity, accelerated by the formalist program of Hilbert and the structuralism of Bourbaki, encouraged a deliberate decoupling. Mathematics became a game of formal symbol manipulation, independent of physical interpretation \cite{hilbert1899}. While logically coherent, this subset of noninterdisciplinary mathematics removes the ``physical'' check. Without the machine (the computer) to replace the physical world as a rigid, non-human verifier, the verification process becomes entirely cognitive and human-dependent.

\section{The Epistemic Meta-Subject and the DID Analogy}
To understand why a community of humans can still possess a ``private'' language, we postulate the existence of the Epistemic Meta-Subject. This entity is the aggregate of all human minds operating within the closed system of noninterdisciplinary pure math.

The analogy to Dissociative Identity Disorder is structural, highlighting the lack of external boundaries:
\begin{enumerate}
    \item \textbf{Internal Multiplicity:} Individual mathematicians act as ``alters'' or distinct personalities within the Meta-Subject. They possess specialized knowledge and communicate with one another.
    \item \textbf{Illusion of Objectivity:} When Mathematician A proposes a proof and Mathematician B reviews it, it appears to be an objective, external check. However, if the axioms and rules of inference are entirely self-referential and decoupled from physical reality, this interaction is topologically equivalent to one alter in a DID system conversing with another.
    \item \textbf{Shared Delusion:} In a DID system, alters can share false memories or delusions. If Alter A believes an event occurred, and Alter B agrees (having access to the same internal script), the agreement does not make the event real. Similarly, if the entire mathematical community agrees on a proof step that is too complex to be physically instantiated or computationally verified, their agreement is merely \emph{internal consistency}, not truth.
\end{enumerate}

Wittgenstein's critique applies because the check is internal to the system (the Meta-Subject). There is no ``thermometer'' outside the body to measure the fever; there is only the body agreeing with itself that it feels hot.

\section{The Role of the Machine: Breaking the Solipsism}
In this framework, the computer (or the physical machine) plays the role of the ``External World.'' 

Machine-aided proofs (such as the verification of the Four Color Theorem \cite{appel1977}) introduce a non-human, rigid agent into the discourse. The machine does not ``share the delusion''; it executes logic gates based on physical laws (electricity, silicon). It serves as the Wittgensteinian external standard.

Without the machine---in our hypothetical scenario---proofs rely solely on human cognitive endurance. Given that human cognition is fallible and suggestible, the ``rules'' of this high-level mathematics lose their rigidity. A proof is accepted not because it is verified against a rigid substrate, but because the Meta-Subject (the community) feels a collective sentiment of correctness. This collapses the distinction between ``being correct'' and ``seeming correct,'' fulfilling the criteria of a private language.

\section{Anticipating Counter-Arguments}
One might argue that formal logic itself constitutes an objective structure, independent of human minds or machines. This is the Platonist defense. However, the PLA is an epistemological critique, not an ontological one. Even if mathematical truths exist independently, our \emph{access} to them via language requires public criteria. 
\begin{itemize}
    \item \emph{Objection:} Peer review works; errors are found.
    \item \emph{Rebuttal:} Un-found errors are un-found.
\end{itemize}

\section{Conclusion}
By applying Wittgenstein's Private Language Argument to the isolated system of noninterdisciplinary pure mathematics, we reveal a startling vulnerability. Without the grounding of empirical science (historical) or computational verification (modern), the global mathematical community risks functioning as a single, dissociated mind. In this state, the distinction between truth and consensus vanishes. The machine, therefore, is not merely a tool for speed; it is the philosophical guarantor of meaning, saving pure mathematics from devolving into the private, unintelligible dream of a solipsistic Meta-Subject.

To conclude, either the DID model challenges Wittgenstein's Private Language Argument, or Wittgenstein's Private Language Argument challenges the entire noninterdisciplinary pure math community.

\bibliographystyle{plain}
\bibliography{references}

\end{document}