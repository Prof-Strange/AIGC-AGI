\documentclass{article}
\usepackage[margin=1in]{geometry}
\usepackage{amsmath}
\usepackage{amsfonts}
\usepackage{amssymb}
\usepackage{hyperref}

\title{Without Machine-Aided Proofs, Modern Pure (Noninterdisciplinary) Mathematics is a Dissociative-Identity-Disorderer's Private Language: A Close Corollary of Wittgensteinian Private Language Argument}
\author{J.L.}
\date{November 19, 2025}

\begin{document}

\maketitle

\begin{abstract}
This paper explores a radical extension of Ludwig Wittgenstein's private language argument (PLA) to the strictly noninterdisciplinary subset of modern pure mathematics--i.e., mathematics deliberately decoupled from any application to empirical science or other disciplines. We argue that, in the absence of machine-aided proof verification tools (such as computers and calculators), this noninterdisciplinary pure mathematics devolves into an ungrounded, private language akin to that of an individual with dissociative identity disorder (DID). By conceptualizing the global community of pure mathematicians as a single ``meta-person'' entity, we demonstrate how the historical decoupling of mathematics from empirical natural sciences renders this subset susceptible to the critiques of the PLA. Without computational aids, the complexity and abstraction of proofs in noninterdisciplinary pure mathematics lack reliable public criteria for correctness. Even if the constituent personalities of the meta-person communicate, the discourse remains internal to the entity and devoid of external checks, rendering it meaningless in Wittgensteinian terms. The analysis is strictly confined to pure noninterdisciplinary mathematics and does not address applied or interdisciplinary mathematics.
\end{abstract}

\section{Introduction}
Ludwig Wittgenstein's private language argument, as presented in his \emph{Philosophical Investigations} \cite{wittgenstein1953}, posits that a language understandable and usable only by a single individual is impossible. Such a language would lack the public rules and criteria necessary for meaningful application, leading to incoherence. Wittgenstein illustrates this with the ``beetle in a box'' thought experiment and the notion of sensation-language, arguing that without communal checks, one cannot distinguish between correct and incorrect usage.

In this paper, we extend the PLA to modern pure mathematics that is explicitly noninterdisciplinary--i.e., the branch concerned with abstract structures, axioms, and theorems deliberately developed without reference to empirical sciences or external applications (e.g., certain areas of number theory, topology, or abstract algebra pursued for their own sake). We contend that, absent machine-aided proofs (hypothetically assuming calculators and computers were never invented), this noninterdisciplinary mathematics resembles a private language spoken by a ``meta-person'' afflicted with dissociative identity disorder (DID). The meta-person is the collective set of all pure mathematicians working in this noninterdisciplinary mode, viewed as a unified entity with multiple ``personalities'' (individual mathematicians) that communicate in an ungrounded idiom.

Historically, pre-Newtonian mathematics was grounded in natural sciences, providing empirical verifiability. However, modern mathematics underwent a deliberate decoupling, accelerated by figures like Hilbert and Bourbaki, elevating a significant subset to a formalist enterprise of pure abstraction with no intended application. This paper examines only that noninterdisciplinary subset.

Our thesis: In the absence of machine aid, the language of noninterdisciplinary pure mathematics becomes a private language--hence meaningless per the PLA--as its ``language'' lacks sufficient public grounding and verifiable criteria.

\section{Wittgenstein's Private Language Argument: A Primer}
Wittgenstein's PLA challenges the Cartesian notion of private mental states as the basis for language. Key points include:

\begin{itemize}
    \item Language requires rules, and rules demand public criteria for following them correctly.
    \item A solitary individual cannot establish such criteria without external checks; memory alone is fallible.
    \item Example: If ``S'' denotes a private sensation, there's no way to confirm if today's ``S'' matches yesterday's without communal agreement or external reference.
\end{itemize}

Applied to mathematics, we ask: Is mathematical truth private or public? In elementary math, it is public. But in advanced noninterdisciplinary pure math, theorems like the Four Color Theorem or the Enormous Theorem involve proofs so intricate that no single human can verify them unaided \cite{appel1977, gorenstein1994}.

\section{The Decoupling of Mathematics from Natural Science}
Pre-Newton, mathematics was intertwined with physics and astronomy, providing empirical grounding. Post-Newton, with the rise of non-Euclidean geometries and abstract algebra, a subset of mathematics deliberately decoupled itself from such grounding. Hilbert's formalist program treated mathematics as a game of symbols with axioms, devoid of required external reference \cite{hilbert1899}. The present paper concerns only this noninterdisciplinary subset that embraces such decoupling.

Without empirical or external anchors, the language of this subset floats freely. Proofs become chains of symbolic manipulations whose verification relies solely on human cognition--fallible and subjective without machine aid.

\section{The Meta-Person and Dissociative Identity Disorder Analogy}
Consider the set of all mathematicians working in noninterdisciplinary pure mathematics as a single meta-person entity. This meta-person ``speaks'' the language of modern noninterdisciplinary pure math through its ``personalities'' (individual mathematicians), each contributing theorems and proofs.

The analogy here is structural rather than pathological. In a DID system, distinct personalities (alters) coexist within one mind. Even if these personalities converse with one another, the dialogue remains entirely internal to the single subject. Similarly:

\begin{itemize}
    \item Mathematicians specialize in niches, serving as distinct alters within the meta-person.
    \item Peer review acts as communication between alters. However, if the entire meta-person (the community) is decoupled from the external world (empirical science) and lacks machine verification, the ``agreement'' reached is merely internal consistency, not public verification.
    \item Without machine aid, verification is like alters sharing memories: ``I feel this proof is correct,'' and another alter agrees. Yet, per Wittgenstein, if the checking mechanism is purely internal to the subject (the meta-person), it is akin to a man buying several copies of the morning newspaper to assure himself that what it said was true.
\end{itemize}

Thus, the meta-person's language is private--internal to its self-contained system--lacking sufficient external grounding. The multiplicity of mathematicians does not constitute a ``public'' in the Wittgensteinian sense, as they collectively form a single, isolated cognitive entity relative to the external physical world.

\section{Without Machine Aid: The Hypothetical Scenario}
Suppose calculators and computers never existed. Human computation is limited: multiplying large numbers or simulating complex systems manually is error-prone.

Proofs in noninterdisciplinary pure mathematics often require:
\begin{itemize}
    \item Exhaustive case-checking (e.g., Four Color Theorem: 1,936 cases \cite{appel1977}).
    \item Symbolic manipulations beyond human endurance.
\end{itemize}

Without machine aid, ``proofs'' become ostensible: a mathematician claims validity, but peers cannot reliably check. This mirrors Wittgenstein's rule-following paradox: ``This is correct because it seems right to me,'' but seeming right isn't a rule \cite{wittgenstein1953}.

Ergo, noninterdisciplinary pure math becomes a private language--meaningless, as per PLA.

\section{Counterarguments and Rebuttals}
One might argue that communal peer review provides public criteria. However, in complex noninterdisciplinary pure math, review is often partial; trust in authority supplants verification. If the entire community is the meta-person, peer review is merely internal monologue.

Another: Math is grounded in logic alone. But without machine checks, logical steps are privately ``felt'' correct.

Applied and interdisciplinary mathematics evade this critique via external tests; hence our strict focus on the noninterdisciplinary subset.

\section{Conclusion}
Extending Wittgenstein's PLA, we conclude that without machine-aided proofs, modern noninterdisciplinary pure mathematics functions as the ungrounded private language of a DID-structured meta-person. This highlights the epistemological role of computation in sustaining abstract discourse within this subset.

\bibliographystyle{plain}
\bibliography{references}

\end{document}